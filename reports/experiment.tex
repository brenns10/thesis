\documentclass{article}

\usepackage[margin=1in]{geometry}
\usepackage{hyperref}
\usepackage{listings}
\usepackage{color}
\definecolor{mygreen}{rgb}{0,0.6,0}
\definecolor{mygray}{rgb}{0.5,0.5,0.5}
\definecolor{mymauve}{rgb}{0.58,0,0.82}
\lstset{%
  backgroundcolor=\color{white},     % background color
  basicstyle=\footnotesize\ttfamily, % monospace, small font
  breaklines=true,                   % nice line breaks
  captionpos=b,                      % put captions at bottom
  commentstyle=\color{mygreen},      % comments
  escapeinside={(*}{*)},             % if you want to add LaTeX in code
  keywordstyle=\color{blue},         % keyword
  stringstyle=\color{mymauve},       % string
}

\title{Proposed Experimentation}
\author{Stephen Brennan}

\begin{document}

\maketitle

\section{Background}

There are several options for deploying networking experiments. In this section,
I will describe several different options, the services they provide, and their
relationships. In particular, I will discuss PlanetLab, GENI, EmuLab, and M-Lab.

\subsection{M-Lab}

\href{https://measurementlab.net}{M-Lab} (more formally Measurement Lab) is a
large-scale, open source measurement partnership. Partners include Google Open
Source Research and PlanetLab itself. M-Lab allows researchers to create
Internet measurement tests that will run on their infrastructure. However, the
requirements for these tests are very restrictive. In particular:

\begin{itemize}
\item Tests must be actively initiated by the client, by visiting the website.
\item Code must be open source.
\item Data must be released into the public domain.
\end{itemize}

Users visit the website and manually run tests (such as Internet speed tests).
As a result, this framework is not likely to be useful for experimentation on
MPTCP, since most users do not run MPTCP capable machines, let alone those with
my customized kernel.

\subsection{PlanetLab}

\href{https://www.planet-lab.org/}{PlanetLab} appears to be the one of the
oldest research frameworks for Internet research. It allows researchers to
create ``slices'' which are containers for resources. Researchers may assign
``nodes'' to their slices, which will give them a virtual server on the slice.
This virtual server is not a true virtual machine - rather it is a lighter
weight containment solution. Although the User Guide suggests that this
mechanism is ``vserver'', more recent announcements suggest that the current
form of virtualization is ``LXC''. In either case, this means that users may not
directly modify the running kernel of their slice. It may be possible to create
a virtual machine on these virtual servers, but the performance impact may
suffer.

Much of the information on PlanetLab's website appears to be out of date or
unmaintained. It is difficult to be certain about what is and is not possible on
the service, when the User Guide itself was last updated in 2006.

\subsection{GENI}

GENI is a newer suite. It has a similar architecture to PlanetLab, organizing
research into slices, nodes, slivers, etc. However, it differs from PlanetLab in
a few key areas. First, its documentation and website are much newer and give a
good deal more information about the services available. Second, it describes
itself as a ``federation'' of computing resource ``aggregates'', including
PlanetLab. In other words, using GENI tools and APIs, researchers may request
and use PlanetLab's resources, along with those of EmuLab and others.

GENI's resources are therefore much more diverse than PlanetLab's. Of course,
PlanetLab slivers would have the same capabilities, regardless of whether you
accessed them through GENI or through PlanetLab directly. However, GENI also
allows access to aggregates which can run custom operating system images. This
would be much more suitable to an experiment which requires a modified kernel.

A downside of GENI is that, although it allows probably the broadest reach in
terms of resource diversity, it may not allow as much control over these
resources as the aggregates themselves might allow. An example is EmuLab, which
will be discussed next. EmuLab resources are likely to have the most flexibility
when requested through EmuLab itself, rather than through GENI. However, I
cannot really be certain about these things until I try them out.

\subsection{EmuLab}

EmuLab describes itself as a ``total network testbed'', giving researchers the
flexibility to emulate arbitrary network topologies. It aims to allow
controllable and repeatable experiments. One of the homepage's main claims is
that it allows you full root access and you may also run custom operating
systems. By all accounts, it seems like the best way to create reliable,
repeatable experiments for this project's use case.

\subsection{Mininet}

Mininet is a single-computer network emulation platform. It allows users to
define arbitrary network topologies, including hosts, links, and switches. The
hosts are simply Linux processes (including potentially child processes) running
in a separate kernel network namespace. All networking occurs on the same
kernel, meaning that the machine running the experiment must have a custom
kernel (rather than simply running a virtual machine with the custom kernel).

It should be possible to run experiments on Mininet - the biggest concern would
be ensuring that the required daemons and kernel modifications don't conflict
with each other, since they would be running on the same kernel with only
network namespaces separating them.

As far as performance, making a Mininet experiment ``network-limited'' requires
that you assign bandwidth to each link that is less than the total your CPU
could handle. On my laptop (within a virtual machine), this maximum is around 24
Gbps. A reasonable setup could assign each link to be 100Mbps, have several
links and routers, and not stress the CPU for processing this.

\section{Experiment Description}

The point of this thesis is to propose a mechanism for applying detour routing
to any TCP connection. While it is out of scope to attempt to determine an
optimal detour routing scheme, some experimentation is necessary, if for no
other reason than to validate that this mechanism can provide the benefits of
Multipath TCP to alternative ``detour'' routes. The main benefit of MPTCP is
that it provides link aggregation, or in this case, path aggregation. That is,
we are able to use the capacity of more than one path in our connection.

Rather than design experiments that attempt to determine the best strategy for
picking a detour, or the best strategy for scheduling data when you have
multiple connections, we should simply show that this implementation provides
these aggregation benefits. There are a couple types of benefits that we should
attempt to quantify:

\begin{itemize}
\item Better throughput. In the case that our connection speed is bottlenecked
  by a link that we can circumvent, we should be able to aggregate bandwidth so
  that we improve throughput.
\item Better response time. In the case that one path has high delay, we should
  be able to improve response time by adding another path.
\item Better fault tolerance. If a path has a high loss rate, we should be able
  to improve our throughput by circumventing that lossy link.
\end{itemize}

So, the experiments are designed around these three scenarios. We use the same
network topologies for each scenario, but we alter the connection properties of
a link along the ``Internet routed path'' (but not the detour routed path).
Then, we compare the desired property (throughput or response time) to the
following controls:

\begin{itemize}
\item Vanilla TCP. No multipath TCP, just a single TCP flow from client to
  server.
\item Vanilla TCP, on the detour route.
\end{itemize}

This way, we can determine whether the MPTCP detour mechanism is able to (at
least) leverage the performance of the better path, if not aggregate their
performance.

These experiments could be run on Mininet or Emulab. In both cases, the results
would be highly reproducible. If Mininet were used, we could run very many
trials very quickly in order to get high statistical power. In addition, for
publication we could provide a complete virtual machine image which could
reproduce all experiments. Finally, if changes were required to the detour
framework (as they will probably be), Mininet will allow for the quickest
response time, so that we can run and re-run our tests quickly.

\end{document}
