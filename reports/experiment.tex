\documentclass{article}

\usepackage[margin=1in]{geometry}
\usepackage{hyperref}
\usepackage{listings}
\usepackage{color}
\definecolor{mygreen}{rgb}{0,0.6,0}
\definecolor{mygray}{rgb}{0.5,0.5,0.5}
\definecolor{mymauve}{rgb}{0.58,0,0.82}
\lstset{%
  backgroundcolor=\color{white},     % background color
  basicstyle=\footnotesize\ttfamily, % monospace, small font
  breaklines=true,                   % nice line breaks
  captionpos=b,                      % put captions at bottom
  commentstyle=\color{mygreen},      % comments
  escapeinside={(*}{*)},             % if you want to add LaTeX in code
  keywordstyle=\color{blue},         % keyword
  stringstyle=\color{mymauve},       % string
}

\title{Proposed Experimentation}
\author{Stephen Brennan}

\begin{document}

\maketitle

\section{Background}

There are several options for deploying networking experiments. In this section,
I will describe several different options, the services they provide, and their
relationships. In particular, I will discuss PlanetLab, GENI, EmuLab, and M-Lab.

\subsection{M-Lab}

\href{https://measurementlab.net}{M-Lab} (more formally Measurement Lab) is a
large-scale, open source measurement partnership. Partners include Google Open
Source Research and PlanetLab itself. M-Lab allows researchers to create
Internet measurement tests that will run on their infrastructure. However, the
requirements for these tests are very restrictive. In particular:

\begin{itemize}
\item Tests must be actively initiated by the client, by visiting the website.
\item Code must be open source.
\item Data must be released into the public domain.
\end{itemize}

Users visit the website and manually run tests (such as Internet speed tests).
As a result, this framework is not likely to be useful for experimentation on
MPTCP, since most users do not run MPTCP capable machines, let alone those with
my customized kernel.

\subsection{PlanetLab}

\href{https://www.planet-lab.org/}{PlanetLab} appears to be the one of the
oldest research frameworks for Internet research. It allows researchers to
create ``slices'' which are containers for resources. Researchers may assign
``nodes'' to their slices, which will give them a virtual server on the slice.
This virtual server is not a true virtual machine - rather it is a lighter
weight containment solution. Although the User Guide suggests that this
mechanism is ``vserver'', more recent announcements suggest that the current
form of virtualization is ``LXC''. In either case, this means that users may not
directly modify the running kernel of their slice. It may be possible to create
a virtual machine on these virtual servers, but the performance impact may
suffer.

Much of the information on PlanetLab's website appears to be out of date or
unmaintained. It is difficult to be certain about what is and is not possible on
the service, when the User Guide itself was last updated in 2006.

\subsection{GENI}

GENI is a newer suite. It has a similar architecture to PlanetLab, organizing
research into slices, nodes, slivers, etc. However, it differs from PlanetLab in
a few key areas. First, its documentation and website are much newer and give a
good deal more information about the services available. Second, it describes
itself as a ``federation'' of computing resource ``aggregates'', including
PlanetLab. In other words, using GENI tools and APIs, researchers may request
and use PlanetLab's resources, along with those of EmuLab and others.

GENI's resources are therefore much more diverse than PlanetLab's. Of course,
PlanetLab slivers would have the same capabilities, regardless of whether you
accessed them through GENI or through PlanetLab directly. However, GENI also
allows access to aggregates which can run custom operating system images. This
would be much more suitable to an experiment which requires a modified kernel.

A downside of GENI is that, although it allows probably the broadest reach in
terms of resource diversity, it may not allow as much control over these
resources as the aggregates themselves might allow. An example is EmuLab, which
will be discussed next. EmuLab resources are likely to have the most flexibility
when requested through EmuLab itself, rather than through GENI. However, I
cannot really be certain about these things until I try them out.

\subsection{EmuLab}

EmuLab describes itself as a ``total network testbed'', giving researchers the
flexibility to emulate arbitrary network topologies. It aims to allow
controllable and repeatable experiments. One of the homepage's main claims is
that it allows you full root access and you may also run custom operating
systems. By all accounts, it seems like the best way to create reliable,
repeatable experiments for this project's use case.

\section{Experiment Description}

The experiments must be designed to answer the important questions about the
detour routing schemes. In particular:

\begin{itemize}
\item Does detour routing improve connection performance when the main route is
  unreliable? How much does it improve performance, and by what performance
  metrics?
\item Is there a measurable performance difference between the OpenVPN and NAT
  approaches to detour routing?
\item Is there a measurable performance difference between scheduling modules in
  this scenario?
\end{itemize}

To quantify performance, the following metrics will be useful:
\begin{itemize}
\item Throughput - total bytes transferred through a connection per unit time.
\end{itemize}

\subsection{Small Scale}

First, we should validate that the framework performs as expected on a
controlled network topology. A reasonable test framework would then be to set up
a network topology with multiple routes between a client and server, vary these
parameters, and compare the performance metrics to those of several controls:

\begin{itemize}
\item Vanilla TCP, using the route which has the ``lossy'' link.
\item Vanilla TCP, using the alternative route through the detour.
\item Multipath TCP, unmodified. In this scenario, the network topology would be
  altered such that the client has two interfaces and each has a different route
  to the server.
\end{itemize}

I would hypothesize that the detour routing solution will outperform the first
option, and it would be comparable to the rest.

% TODO: diagram of proposed topology

\subsection{Large Scale}

My ideas of what would make a suitable larger scale test are not as well-formed.
I believe we need a larger scale test to give some indication that this approach
could work ``in the wild''. For this sort of test, rather than setting up a
controlled, repeatable network topology, we would simply use nodes in separate
sites and perform a similar data transfer. The transfer would have to be
repeated (control and MPTCP) several times in a row to give real statistical
backbone to the experiment.


\end{document}
