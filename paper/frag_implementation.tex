\section{Implementation}
% Section intro:
% 1. Expectations (what hosts are there, and which ones run our code?)
% 2. Overview of the distinct pieces of software (client kernel, client daemon,
%    detour daemon)
% Subsection - Detour daemon
% Subsection - Client kernel
% Subsection - Client daemon

% smoother transition once we have everything

The implementation of this project is based on the following expectations.
First, the server must support Multipath TCP. However, beyond this, no other
modifications should be required of the server. Second, the client and detour
point are assumed to be ``active'' participants in the collective, and so they
may run whatever software is necessary to participate.

The client must run a modified kernel, which supports creating MPTCP subflows
across detours. In addition, the client must also run a daemon which discovers
and negotiates with detour points, finally reporting available ones to the
kernel. These concerns remain separate for several reasons. First, communication
with detour points requires communication over the Internet. While it is
possible to do user-level networking within the kernel, it is much simpler and
more future-proof to use the userspace socket API. Second, one type of detour
requires starting an OpenVPN process, an operation which should only be done
from userspace. Finally, userspace tools may be more easily configured and can
perform a more broad spectrum of computations, such as floating point math.

The detour need not support Multipath TCP. However, it must run one of two
possible daemons, depending on which type of detour it will be hosting.

\subsection{Detour Daemon}

In this project, we implement two different mechanisms for tunneling MPTCP
subflows across a detour. In both cases, the detour host must run a daemon that
assists in tunneling traffic from the client to the server.

\subsubsection{OpenVPN Tunneling}

The first mechanism involves the open-source tool OpenVPN. Detours run a
specially configured OpenVPN server. Clients connect to this VPN, creating a
virtual network device. The client receives a routing rule from the detour
advertising that it can reach all hosts on the Internet with a very high cost.
As a result, the client operating system will prefer other routes to the VPN.

The VPN connection is configured as follows:

\begin{itemize}
\item The connection is over UDP, to avoid the so-called ``TCP Meltdown'' effect
  caused by two congestion control algorithms interfering with each other.
  % TODO citation
\item The initial connection negotiation requires the client and server to
  exchange certificates. OpenVPN allows the user to specify the PKI root. A
  centralized detour management system could authorize clients and servers by
  signing certificates, enabling tunneling for members and preventing it for
  non-members.
\item Subsequent messages are not protected by encryption or signatures, to
  avoid the computational overhead.
\end{itemize}

One important aspect of VPN configuration is that the detour must create a
private IP subnet. It will provide a DHCP service in order to assign IP
addresses to clients as they connect to the VPN. Clients are expected to be able
to connect to multiple detours simultaneously. In order to do this without
conflict, each detour must use a distinct private IP subnet, so that there is no
chance of the client addresses overlapping, and also so that the gateway address
of the VPN is unique on the client.

In a large-scale implementation of this technology, these subnet allocations
could be handled by a centralized detour management server. The 10.0.0.0/8
subnet is reserved for private networks, and if each detour were to establish
its own /24 subnet, there would be capacity for 65,536 non-conflicting subnets.
However, in our testing, subnets were manually allocated to avoid conflicts.

Finally, in order to forward VPN traffic to the internet, the detour must be
configured (via Netfilter) to perform network address translation (NAT) on
outgoing packets. This process is typical for a home router and for VPNs which
provide Internet access. Multipath TCP is designed with this behavior in mind,
and so it is capable of functioning through NAT, so long as the middleboxes do
not remove TCP options in transit.

% TODO cite MPTCP design/architecture for middleboxes

The OpenVPN implementation has several attractive qualities. First, it relies on
a well-used protocol for forwarding traffic. Second, it need not be configured
on a per-connection basis. Finally, it provides a built in mechanism for
mutually authenticating the client and detour. However, there are some
drawbacks. Since packets are tunneled, at least 28 bytes of overhead (IP and UDP
headers) are required. To avoid fragmentation, the connection's maximum segment
size (MSS) must be adjusted.

\subsubsection{NAT Tunneling}

The second approach directly modifies packets, without actually using a protocol
(like OpenVPN) for tunneling. Since we only need to forward some connections to
some destinations, the full power of OpenVPN is not necessary. Instead, packets
are addressed directly to the detour. The detour can forward these to the
final destination, and forward reply packets to the client. In order to do this,
the detour must have advance knowledge of the final destination of the packet.
So, clients use a simple signaling protocol to request tunnels via a detour.

The Linux kernel's Netfilter framework allows for two types of NAT. The first
kind is source NAT, or SNAT, rewrites the source address of the packet. This is
the typical form of NAT used by home routers: the source address of an outgoing
connection is rewritten to be the router's external IP address, but the
destination is unchanged. The second kind of NAT supported by Netfilter is
destination NAT, or DNAT. This rewrites the destination address of the packet.

% TODO - netfilter docs citation

The detour takes advantage of both forms of NAT offered by the kernel. First, it
applies DNAT, setting the destination address to be the final server's IP
address (arranged ahead of time). Then, it applies SNAT, setting the source
address to be its own external IP. The Netfilter framework remembers these
connections so that reply packets are properly forwarded back to the client.

Thanks to Netfilter, this NAT mapping can be arranged with two \texttt{iptables}
commands, with no custom kernel or user-space code required for packet
forwarding. It is possible to implement this in userspace using raw sockets, but
this approach has several drawbacks. It would have to be single-threaded, and it
would require that packet data be copied from kernel space into user space and
back. More critically, the connection tracking provided by the kernel is more
robust than what would be implemented by user space.

As mentioned earlier, the server address must be communicated to the detour
ahead of time in order to use this scheme. To that end, we have designed a
simple UDP-based for clients to ``request'' NAT mappings from a detour. The
detour listens for requests on UDP port 45672. The request format is shown
below.

\begin{lstlisting}
+-----------------------------------+
| ver(1) | op(1)  | reserved (2)    |
+-----------------------------------+
|           rip (4 bytes)           |
+-----------------------------------+
| rpt (2 bytes)   | dpt (2 bytes)   |
+-----------------|-----------------+
\end{lstlisting}

The \texttt{ver} field is a protocol version, currently set to 1. The
\texttt{op} field contains the operation of this message: request (0) or
response (1). The \texttt{reserved} field is unused, currently existing to pad
the request. The field \texttt{rip} (short for ``remote IP'') is the address
that the client wishes to create a detour to. \texttt{rpt} is the TCP port on
the server to create a detour to. To understand the \texttt{dpt} field, it is
useful to describe all of the addresses and ports in use during a tunneled
connection:
\begin{itemize}
\item Client IP, port: the IP address of the client, and the client-side port,
  which is usually ephemeral.
\item Detour IP, port (client side): the IP and port that the client will
  address its communication to.
\item Detour IP, port (remote side): the IP and port which are used when the
  detour opens its corresponding connection on the remote.
\item Remote IP, port: the IP and port which the client is tunneling its
  connection to.
\end{itemize}

The \texttt{dpt} field corresponds to the detour port, client side. In a
request, the client proposes a port. In the response, the detour specifies the
actual port which will be used. The server may wish to alter the \texttt{dpt}
field for several reasons:
\begin{itemize}
\item A tunnel to the remote IP, port pair is already open on a different detour
  port.
\item The client already has a detour open (to a different remote) on the given
  port.
\item The detour may have a socket accepting connection on the proposed port,
  and so it may wish to avoid creating a tunnel which would prevent the client
  from accessing that service.
\end{itemize}

In order to create the tunnels, the detour runs two IPTables commands, which
create NAT rules.
