\section{Background}
% Potential additions:
% - netlink? generic netlink?

\subsection{Multipath TCP}
% Our discussion of Multipath TCP needs to be in enough depth that the reader
% understands how it works - essentially covering at least the first few pages
% of the RFC. In particular:
% - Subflows: what they are and how they work
% - How the initial subflow is created
% - How subsequent subflows are created and authenticated
% - Data sequencing and how data sequences are mapped onto subflow sequences
% - path management and data scheduling
% - ? how the linux kernel implementation is structured

In recent years, multi-homed devices have become increasingly common. The most
obvious example of these devices is the smartphone, which typically comes with
at least two network interfaces: one for cellular data, and one for Wi-Fi. While
these devices have become more common, most Internet protocols do not have ways
to take advantage of multiple interfaces.

Multipath TCP (MPTCP) is an extension to TCP which aims to mitigate this
problem. It is designed with several goals in mind:

\begin{itemize}
  \item To improve both the throughput and reliability of connections, relative
    to normal TCP.
  \item To remain compatible with applications that use TCP, so that they could
    use MPTCP without modification.
  \item To remain compatible with the Internet as a whole, especially with
    ``middleware'' such as firewalls, proxies, and NATs.
\end{itemize}
% https://tools.ietf.org/html/rfc6182#section-2

It achieves these goals by layering the protocol b

\subsection{Netfilter}
% Discuss what it is, and how it can be used to perform NAT

\subsection{OpenVPN}
% This will have to at least touch on TUN/TAP devices, and the variations of the
% OpenVPN protocol.
